% ***************************************************
% FORMATO.
% ***************************************************
\documentclass[a4paper,12pt]{article}
% Input encoding.
% Idioma.
\usepackage[spanish, es-tabla]{babel}
\usepackage[utf8]{inputenc}
% Ver pagina 92 de "The not so short introduction to Latex" por Tobias Oetiker
% para entender que hacen estos paquetes.
\usepackage{lmodern}
\usepackage[T1]{fontenc}
\usepackage{textcomp}
% Paquete para utilizar el font Times Roman de Adobe por defecto.
\usepackage{mathptmx}
 
%Manejo de márgenes
%Agrega 2cm al ancho por defecto (usado junto con hoffset, no se pierde el centrado)
\addtolength{\textwidth}{2cm}
%Quita 1cm al margen, pero mantiene la relación repecto de cada margen
\addtolength{\hoffset}{-1cm}
\addtolength{\textheight}{2cm}
\addtolength{\voffset}{-1cm}
\setlength{\headheight}{14pt}

% Paquete para manejo de headers y footers.
\usepackage{fancyhdr}
\usepackage{lastpage}
% Determinación del estilo de página (encabezado y pie). Debe ir luego del bloque de manejo de margenes para que
% no haya problemas de dimensiones. Por ejemplo, que el texto quede mas ancho que el encabezado o pie.
\pagestyle{fancy}
% Se borra el header y footer por defecto.
\fancyhf{}
\lhead{\small TP1 [2C2018]}
\chead{}
\rhead{\small [66.20] Organización de Computadoras}
\rfoot{}
\cfoot{Página {\thepage} de \pageref{LastPage}}
\lfoot{}

% Paquete para manejo de textos en varbatim en varios lugares del documento, por ejemplo en los headers.
\usepackage{fancyvrb}
% Paquete para manejo de boxes de manera mas flexible.
\usepackage{fancybox}

% Para no mostrar el numero de pagina, usar las siguientes dos lineas:
%\thispagestyle{empty}
%\setcounter{savepage}{\thepage}

% Paquete para hacer subrayados, poner textos en color y resaltar texto en color.
% Lo bueno de este paquete es que no tira errores de fullbox como cuando se usa \colorbox{declared-color}{text}.
% El unico problema es que no soporta acentos desde el teclado (tira un error de UTF8). Hay que incluirlos usando \'.
\usepackage{soul}
% Configura el paquete @SOUL para que el resaltado de texto sea rojo. La sintaxis es \hl{text}.
\sethlcolor{red}

% Paquete que deja sangria luego de comenzada una seccion nueva.
%\usepackage{indentfirst}

% Se define una macro para poner double quotes de manera simple. El uso es \quotes{Hello World!}.
\newcommand{\quotes}[1]{``#1''}

% ***************************************************
% MATEMATICA.
% ***************************************************
\usepackage{amsmath}
	% Formato de numeracion de ecuaciones por seccion. Ej. seccion x, ecuacion y: (x.y)
	\numberwithin{equation}{section}
	\numberwithin{figure}{section}
	
% Simbolos matematicos. Por ej. la 'R' de reales, etc...
\usepackage{amssymb}
\usepackage{amsfonts}

\usepackage{mathtools}

% Paquete para arrays de ecuaciones.
\usepackage[retainorgcmds]{IEEEtrantools}
% Unidad imaginaria.
\newcommand{\iu}{{j\mkern1mu}}
% Contador de uso generico.
\newcounter{hipotesis_ct}
\setcounter{hipotesis_ct}{1}

% Operador 'derivada'.
\makeatletter
\providecommand*{\diff}%
{\@ifnextchar^{\DIfF}{\DIfF^{}}}
\def\DIfF^#1{%
	\mathop{\mathrm{\mathstrut d}}%
	\nolimits^{#1}\gobblespace}
\def\gobblespace{%
	\futurelet\diffarg\opspace}
\def\opspace{%
	\let\DiffSpace\!%
	\ifx\diffarg(%
	\let\DiffSpace\relax
	\else
	\ifx\diffarg[%
	\let\DiffSpace\relax
	\else
	\ifx\diffarg\{%
	\let\DiffSpace\relax
	\fi\fi\fi\DiffSpace}

% Uso:
% \pderiv[n]{A}{B}
\providecommand*{\deriv}[3][]{%
	\frac{\diff^{#1}#2}{\diff #3^{#1}}}
\providecommand*{\pderiv}[3][]{%
	\frac{\partial^{#1}#2}%
	{\partial #3^{#1}}}

% Operadores 'valor absoluto' y 'norma'.
% Uso: \abs{123} \norm{123} \abs*{123} \norm*{123}
\DeclarePairedDelimiter\abs{\lvert}{\rvert}%
\DeclarePairedDelimiter\norm{\lVert}{\rVert}%
% Swap the definition of \abs* and \norm*, so that \abs
% and \norm resizes the size of the brackets, and the 
% starred version does not.
\makeatletter
\let\oldabs\abs
\def\abs{\@ifstar{\oldabs}{\oldabs*}}
%
\let\oldnorm\norm
\def\norm{\@ifstar{\oldnorm}{\oldnorm*}}
\makeatother

% Operadores para escribir vectores en negrita directamente.
% Ej:  $\vecbf{v} = 5\vechat{k}$ 
\newcommand{\vecbf}[1]{\mathbf{#1}}
% Operador para escribir en negrita todo lo que tenga hat.
\newcommand{\vechat}[1]{\hat{\mathbf{#1}}}

% ***************************************************
% CODIGO FUENTE.
% ***************************************************
% Paquetes para pseudocodigo.	
\usepackage{lgrind}
\usepackage{algorithm}
\usepackage{algpseudocode}
\makeatletter
\renewcommand{\ALG@name}{Algoritmo}
\renewcommand{\listalgorithmname}{Lista de \ALG@name s}
\algrenewcommand\algorithmicensure{\textbf{Salida:}}
\algrenewcommand\algorithmicrequire{\textbf{Entrada:}}
\algnewcommand{\LineComment}[1]{\State \(\triangleright\) #1}
\makeatother

% Paquete para incluir codigo de MatLab.
\usepackage[numbered,framed]{matlab-prettifier}
% Paquete para control de colores. Ya lo carga 'matlab-prettifier'.
\PassOptionsToPackage{usenames,dvipsnames}{xcolor}

% Fuente: http://en.wikibooks.org/wiki/LaTeX/Colors
\definecolor{Apricot}{RGB}{253,199,130}
\definecolor{Aquamarine}{RGB}{0,181,202}
\definecolor{Bittersweet}{RGB}{192,79,23}
\definecolor{Black}{RGB}{0,0,0}
\definecolor{Blue}{RGB}{0,0,255}
\definecolor{BlueGreen}{RGB}{0,179,184}
\definecolor{BlueViolet}{RGB}{71,57,146}
\definecolor{BrickRed}{RGB}{182,50,28}
\definecolor{Brown}{RGB}{121,37,0}
\definecolor{BurntOrange}{RGB}{247,146,29}
\definecolor{CadetBlue}{RGB}{116,114,154}
\definecolor{CarnationPink}{RGB}{242,130,180}
\definecolor{Cerulean}{RGB}{0,162,227}
\definecolor{CornflowerBlue}{RGB}{65,176,228}
\definecolor{Cyan}{RGB}{0,174,239}
\definecolor{Dandelion}{RGB}{253,188,66}
\definecolor{DarkOrchid}{RGB}{164,83,138}
\definecolor{Emerald}{RGB}{0,169,157}
\definecolor{ForestGreen}{RGB}{0,155,85}
\definecolor{Fuchsia}{RGB}{140,54,140}
\definecolor{Goldenrod}{RGB}{255,223,66}
\definecolor{Gray}{RGB}{148,150,152}
\definecolor{Green}{RGB}{0,166,79}
\definecolor{GreenYellow}{RGB}{223,230,116}
\definecolor{JungleGreen}{RGB}{0,169,154}
\definecolor{Lavender}{RGB}{244,158,196}
\definecolor{LimeGreen}{RGB}{141,199,62}
\definecolor{Magenta}{RGB}{236,0,140}
\definecolor{Mahogany}{RGB}{169,52,31}
\definecolor{Maroon}{RGB}{175,50,53}
\definecolor{Melon}{RGB}{248,158,123}
\definecolor{MidnightBlue}{RGB}{0,103,149}
\definecolor{Mulberry}{RGB}{169,60,147}
\definecolor{NavyBlue}{RGB}{0,103,149}
\definecolor{OliveGreen}{RGB}{60,128,49}
\definecolor{Orange}{RGB}{245,129,55}
\definecolor{OrangeRed}{RGB}{237,19,90}
\definecolor{Orchid}{RGB}{175,114,176}
\definecolor{Peach}{RGB}{247,150,90}
\definecolor{Periwinkle}{RGB}{121,119,184}
\definecolor{PineGreen}{RGB}{0,139,114}
\definecolor{Plum}{RGB}{146,38,143}
\definecolor{ProcessBlue}{RGB}{0,176,240}
\definecolor{Purple}{RGB}{153,71,155}
\definecolor{RawSienna}{RGB}{151,64,6}
\definecolor{Red}{RGB}{237,27,35}
\definecolor{RedOrange}{RGB}{242,96,70}
\definecolor{RedViolet}{RGB}{161,36,107}
\definecolor{Rhodamine}{RGB}{239,85,159}
\definecolor{RoyalBlue}{RGB}{0,113,188}
\definecolor{RoyalPurple}{RGB}{97,63,153}
\definecolor{RubineRed}{RGB}{237,1,125}
\definecolor{Salmon}{RGB}{246,146,137}
\definecolor{SeaGreen}{RGB}{63,188,157}
\definecolor{Sepia}{RGB}{103,24,0}
\definecolor{SkyBlue}{RGB}{70,197,221}
\definecolor{SpringGreen}{RGB}{198,220,103}
\definecolor{Tan}{RGB}{218,157,118}
\definecolor{TealBlue}{RGB}{0,174,179}
\definecolor{Thistle}{RGB}{216,131,183}
\definecolor{Turquoise}{RGB}{0,180,206}
\definecolor{Violet}{RGB}{88,66,155}
\definecolor{VioletRed}{RGB}{239,88,160}
\definecolor{White}{RGB}{255,255,255}
\definecolor{WildStrawberry}{RGB}{238,41,103}
\definecolor{Yellow}{RGB}{255,242,0}
\definecolor{YellowGreen}{RGB}{152,204,112}
\definecolor{YellowOrange}{RGB}{250,162,26}


\lstdefinestyle{StyleMake}{%
	linewidth=\textwidth,%		Define el ancho máximo de una linea de código
	xleftmargin=2.5pt,%				Margen izquierdo
	xrightmargin=2.5pt,%				Margen derecho
	breaklines=true,%				Que corte lineas largas
	numbers=left,%			Que haya números a la izquierda (número de linea)
	numberstyle=\scriptsize,%		Formato de los números de linea
	stepnumber=1,%					Que avancen de a 1 (ver todos los números)
	numbersep=10pt,%					La separación de los números del código
	tabsize=4,showtabs=false,%		Los TABS son 4 espacios, y no se ven
	extendedchars=true,%			Se permiten caracteres extendidos
	basicstyle=\ttfamily\small,%	Estilo básico de la tipografía del código
	commentstyle=\color{Blue},%		Color de los comentarios
	showstringspaces=false,%		Los espacios no se ven de forma especial
	stringstyle=\color{BrickRed},%	Estilo de las cadenas
	keywordstyle=\color{ForestGreen},%	Estilo de las palabras reservadas
	morekeywords=[1]{g++,-Wall,-pedantic},%	Definición de más palabras reservadas
	deletekeywords={rm,-rf,*.o},%
	keywords=[2]{rm,-rf,*.o},%
	keywordstyle=[2]{\color{BurntOrange}},%	2do grupo de palabras reservadas
	directives={CC,CFLAGS,OUTPUT},%
	directivestyle=\color{NavyBlue},%	Estilo de directivas
	keywords=[3]{CC,CFLAGS,OUTPUT},%
	keywordstyle=[3]{\color{MidnightBlue}},%3er grupo de palabras reservadas
	morecomment=[s][\color{Blue}]{/*}{*/}%	Definición de estilo de comentario
}

\lstdefinestyle{StyleC}{%
	linewidth=\textwidth,%		Define el ancho máximo de una linea de código
	xleftmargin=2.5pt,%				Margen izquierdo
	xrightmargin=2.5pt,%				Margen derecho
	breaklines=true,%				Que corte lineas largas
	numbers=left,%			Que haya números a la izquierda (número de linea)
	numberstyle=\scriptsize,%		Formato de los números de linea
	stepnumber=1,%					Que avancen de a 1 (ver todos los números)
	numbersep=10pt,%					La separación de los números del código
	tabsize=4,showtabs=false,%		Los TABS son 4 espacios, y no se ven
	extendedchars=true,%			Se permiten caracteres extendidos
	basicstyle=\ttfamily\small,%	Estilo básico de la tipografía del código
	commentstyle=\color{Blue},%		Color de los comentarios
	showstringspaces=false,%		Los espacios no se ven de forma especial
	stringstyle=\color{BrickRed},%	Estilo de las cadenas
	keywordstyle=\color{ForestGreen},%	Estilo de las palabras reservadas
	morekeywords=[1]{size_t,ssize_t},%	Definición de más palabras reservadas
	deletekeywords={typedef,enum,do,while,if,else,for,case,default,switch,break,continue},%
	keywords=[2]{typedef,enum,do,while,if,else,for,case,default,switch,break,continue,EXIT_SUCCESS,EXIT_FAILURE},%
	keywordstyle=[2]{\color{BurntOrange}},%	2do grupo de palabras reservadas
	directives={define,undef,include,if,else,ifndef,ifdef,elif,endif},%
	directivestyle=\color{NavyBlue},%		Estilo de directivas
	keywords=[3]{define,undef,include,if,else,ifndef,ifdef,elif,endif},%
	keywordstyle=[3]{\color{MidnightBlue}},%3er grupo de palabras reservadas
	morecomment=[s][\color{Blue}]{/*}{*/}%	Definición de estilo de comentario
}

\lstnewenvironment{terminalblock}{%
	\lstset{style=StyleC}}{}

\lstset{
	literate={ö}{{\"o}}1
	{ä}{{\"a}}1
	{ü}{{\"u}}1
	{á}{{\'a}}1
	{Á}{{\'A}}1
	{é}{{\'e}}1
	{É}{{\'E}}1
	{í}{{\'i}}1
	{Í}{{\'I}}1
	{ñ}{{\~n}}1
	{ó}{{\'o}}1
	{Ó}{{\'O}}1
	{ú}{{\'u}}1
	{Ú}{{\'U}}1
}

\definecolor{listinggray}{gray}{0.9}
\definecolor{lbcolor}{rgb}{0.9,0.9,0.9}
% http://tex.stackexchange.com/questions/44282/multiline-comment
\newcommand{\comment}[1]{}

\lstset{
	backgroundcolor=\color{lbcolor},
	tabsize=4,    
	language=[GNU]C++,
	basicstyle=\scriptsize,
	upquote=true,
	aboveskip={1.5\baselineskip},
	columns=fixed,
	showstringspaces=false,
	extendedchars=false,
	breaklines=true,
	prebreak = \raisebox{0ex}[0ex][0ex]{\ensuremath{\hookleftarrow}},
	frame=single,
	numbers=left,
	showtabs=false,
	showspaces=false,
	showstringspaces=false,
	identifierstyle=\ttfamily,
	keywordstyle=\color[rgb]{0,0,1},
	commentstyle=\color[rgb]{0.026,0.112,0.095},
	stringstyle=\color[rgb]{0.627,0.126,0.941},
	numberstyle=\color[rgb]{0.205, 0.142, 0.73},
}

\lstset{
	backgroundcolor=\color{lbcolor},
	tabsize=4,
	language=C++,
	captionpos=b,
	tabsize=3,
	frame=lines,
	numbers=left,
	numberstyle=\tiny,
	numbersep=5pt,
	breaklines=true,
	showstringspaces=false,
	basicstyle=\footnotesize,
	keywordstyle=\color[rgb]{0,0,1},
	commentstyle=\color[rgb]{0.026,0.112,0.095},
	stringstyle=\color{red},
}
% ***************************************************
% GRAFICOS.
% ***************************************************
	% Paquete para mejorar la interfaz con objetos flotantes, como los graficos.
	\usepackage{float}
	\usepackage{graphicx}
	% Paquetes para tener mayor flexibilidad cuando se utilizan titulos para figuras, etc...
	\usepackage{caption}
	\usepackage{subcaption}
	\usepackage{epstopdf}

% ***************************************************
% TABLAS.
% ***************************************************
% Paquetes para manejo de tablas.
\usepackage{booktabs}
\usepackage{multirow}

% ***************************************************
% MISC.
% ***************************************************
%\usepackage{draftwatermark}
%\SetWatermarkLightness{0.95}
%\SetWatermarkScale{0.8}
%\SetWatermarkText{BORRADOR}

% This pack­age sim­pli­fies the in­clu­sion of ex­ter­nal multi-page PDF doc­u­ments in LaTeX doc­u­ments. 
\usepackage{pdfpages}
\setboolean{@twoside}{false}

% Unidades del SI
\usepackage{siunitx}

% Adds bibliography section to the TOC.
\usepackage[nottoc,notlot,notlof]{tocbibind}

% ***************************************************
% TIKZ.
% ***************************************************
%\usepackage{tikz}
%\usetikzlibrary{calc,fadings,decorations.pathreplacing}
%%% helper macros
%\newcommand\pgfmathsinandcos[3]{%
%	\pgfmathsetmacro#1{sin(#3)}%
%	\pgfmathsetmacro#2{cos(#3)}%
%}
%\newcommand\LongitudePlane[3][current plane]{%
%	\pgfmathsinandcos\sinEl\cosEl{#2} % elevation
%	\pgfmathsinandcos\sint\cost{#3} % azimuth
%	\tikzset{#1/.estyle={cm={\cost,\sint*\sinEl,0,\cosEl,(0,0)}}}
%}
%\newcommand\LatitudePlane[3][current plane]{%
%	\pgfmathsinandcos\sinEl\cosEl{#2} % elevation
%	\pgfmathsinandcos\sint\cost{#3} % latitude
%	\pgfmathsetmacro\yshift{\cosEl*\sint}
%	\tikzset{#1/.estyle={cm={\cost,0,0,\cost*\sinEl,(0,\yshift)}}} %
%}
%\newcommand\DrawLongitudeCircle[2][1]{
%	\LongitudePlane{\angEl}{#2}
%	\tikzset{current plane/.prefix style={scale=#1}}
%	% angle of "visibility"
%	\pgfmathsetmacro\angVis{atan(sin(#2)*cos(\angEl)/sin(\angEl))} %
%	\draw[current plane] (\angVis:1) arc (\angVis:\angVis+180:1);
%	\draw[current plane,dashed] (\angVis-180:1) arc (\angVis-180:\angVis:1);
%}
%\newcommand\DrawLatitudeCircle[2][1]{
%	\LatitudePlane{\angEl}{#2}
%	\tikzset{current plane/.prefix style={scale=#1}}
%	\pgfmathsetmacro\sinVis{sin(#2)/cos(#2)*sin(\angEl)/cos(\angEl)}
%	% angle of "visibility"
%	\pgfmathsetmacro\angVis{asin(min(1,max(\sinVis,-1)))}
%	\draw[current plane] (\angVis:1) arc (\angVis:-\angVis-180:1);
%	\draw[current plane,dashed] (180-\angVis:1) arc (180-\angVis:\angVis:1);
%}
%
%%% document-wide tikz options and styles
%\tikzset{%
%	>=latex, % option for nice arrows
%	inner sep=0pt,%
%	outer sep=2pt,%
%	mark coordinate/.style={inner sep=0pt,outer sep=0pt,minimum size=3pt,
%		fill=black,circle}%
%}

% ***************************************************
% HIPERTEXTO.
% ***************************************************
% Paquete para utilizar hipertexto (tiene que ser siempre el ultimo paquete que se carga).
% Todos los links, referencias, etc...pasan a ser hipertextos.
% Setup del paquete de hipertexto.
\usepackage[pdftex,breaklinks]{hyperref}
	\hypersetup{a4paper,
				% Muestra el titulo en el maro superior de la ventana del visor de pdf
				pdfdisplaydoctitle=true,
				% Al encender esta funcion, ya nohay recuadros feos alrededor de los links 	
				colorlinks=true,
				% Color cuando se escribe una URL.   
				urlcolor = black,
				%Para que el link a nuevos pdf, no se abran en la misma ventana 		
				pdfnewwindow=true,
				% Color para links internos
				linkcolor=black,
				% Color para links a Bibliografia
				citecolor=black,
				% Color para links a archivos
       			filecolor=black,
				pdftitle={66.20 - TP1 - 2C2018},
				pdfauthor={AHV}
			}

% ***************************************************			
% GLOSARIO.
% ***************************************************
% Glosario. Debe cargarse después del paquete "hyperref", como excepción.
%\include{./misc/glossary}

% ***************************************************			
% ALGUNOS TEMPLATES DE CODIGO UTIL.
% ***************************************************

% *********************************************************************************
% Si incluyo figuras de MatLab, no conviene que se guarden maximizadas.
%\begin{figure}[!ht]
%	\centering
%	\includegraphics[scale=0.9]{includes/ex1_N5.eps}
%	\caption{Submuestreo con $N=5$ con $M=3$ y $M=6$}
%	\label{ex1_N5}
%\end{figure}
% *********************************************************************************
%\begin{figure}[H]
%	\centering
%	\begin{subfigure}[c]{0.3\textwidth}
%		\includegraphics[scale=0.06]{./includes/cover/Logo_UBA.eps}
%	\end{subfigure}
%	\begin{subfigure}[c]{0.3\textwidth}
%		\includegraphics[scale=0.5]{./includes/cover/Logo_FIUBA.eps}
%	\end{subfigure}
%\end{figure}
% *********************************************************************************
%\begin{IEEEeqnarray}{rCl}
%\IEEEyesnumber\label{ec_ex1_relaciones} \IEEEyessubnumber*
%x\left( n \right)  \  &=& \, x\left( n + k \, N \right) \label{ec_ex1_x}\\
%y\left( n \right)  \, &=& \, y\left( n + l \, L \right) \label{ec_ex1_y}\\
%y\left( n \right)  \, &=& \, x\left( M \, n \right) \label{ec_ex1_downsampling}
%\end{IEEEeqnarray}
%\begin{IEEEeqnarray}{CC}
%	\IEEEyesnumber\label{eq:both} \IEEEyessubnumber*
%	bla bla & blub blub \label{eq:sub1}\\
%	bla bla & bla bla  \label{eq:sub2}
%\end{IEEEeqnarray}
%where the set of equations \eqref{eq:sub1} and \eqref{eq:sub2}
%is referred to as \eqref{eq:both}.
%\begin{align}
%N_{x,sk} &= k_{sk}\left(\frac{t_{sk}}{b_{sk}}\right)^{2}\bar{Et}\nonumber\\
%N_{x,st} &= k_{st}\left(\frac{t_{st}}{b_{st}}\right)^{2}\bar{Et}
%\end{align}
% *********************************************************************************
%\begin{equation}
%\begin{cases}
%H\left( z \right) &= \sum_{l=0}^{M-1} \, z^{-l} \, E_l\left( z^M \right) \\
%E_l\left( z \right) &= \sum_{n=-\infty}^{+\infty} \, e_l \left( n \right) \, z^{-n} \\
%e_l\left( n \right) &= h\left(n\,M + l\right) \quad 0 \leq l \leq M-1
%\end{cases}
%\end{equation}
% *********************************************************************************
%\begin{equation}
%\mathbf{R}_x\left(\phi\right) = 
%\begin{bmatrix}
%1 & 0 & 0 \\
%0 & \cos \left(\phi\right) & \sin \left(\phi\right) \\
%0 & -\sin \left(\phi\right) & \cos \left(\phi\right) \\
%\end{bmatrix}
%\end{equation}
% *********************************************************************************
%\begin{algorithm}[!ht]
%\caption{Pseudocódigo del algoritmo euclídeo extendido}\label{alg_euc_ex}
%\begin{algorithmic}[1]
%\Require $H_0\left(z\right)$ y $H_1\left(z\right)$ coprimos
%\Ensure  $F_0\left(z\right)$ y $F_1\left(z\right)$ tales que $H_0\left(z\right)\cdot F_0\left(z\right)+H_1\left(z\right)\cdot F_1\left(z\right) = c$, con $c\neq0$. Además, $gr\left(F_0\right) < gr\left(H_1\right) - gr\left(g\right)$ y $gr\left(F_1\right) < gr\left(H_0\right) - gr\left(g\right)$.
%\Statex
%\Procedure{extEuclidean}{}
%\Statex
%\State Inicialización.
%\Statex
%\State $r_0=H_0 \quad r_1=H_1 \quad s_0=1 \quad s_1=0 \quad t_0=0 \quad t_1=1 \quad i=1$
%\Statex
%\While {$r_i\neq0 \quad\wedge\quad gr\left(r_{i+1}\right)\geqslant gr\left(r_i\right)$:}
%\Statex
%\State $q \leftarrow$ Cociente de $r_{i-1}/r_i$
%\Statex
%\State $r_{i+1} = r_{i-1} - q\,r_i$
%\Statex
%\State $s_{i+1} = s_{i-1} - q\,s_i$
%\Statex
%\State $t_{i+1} = t_{i-1} - q\,t_i$
%\Statex
%\State $i = i+1$
%\EndWhile
%\Statex
%\State $F_0\left(z\right) = s_{i-1} \quad F_1\left(z\right)=t_{i-1} \quad g=r_{i-1}$
%\EndProcedure
%\end{algorithmic}
%\end{algorithm}
% *********************************************************************************
%\begin{table}
%	\caption{A badly formatted table}
%	\centering
%	\label{table:bad_table}
%	\begin{tabular}{|l|c|c|c|c|}
%		\hline 
%		& \multicolumn{2}{c}{Species I} & \multicolumn{2}{c|}{Species II} \\ 
%		\hline
%		Dental measurement  & mean & SD  & mean & SD  \\ \hline 
%		\hline
%		I1MD & 6.23 & 0.91 & 5.2  & 0.7  \\
%		\hline 
%		I1LL & 7.48 & 0.56 & 8.7  & 0.71 \\
%		\hline 
%		I2MD & 3.99 & 0.63 & 4.22 & 0.54 \\
%		\hline 
%		I2LL & 6.81 & 0.02 & 6.66 & 0.01 \\
%		\hline 
%		CMD & 13.47 & 0.09 & 10.55 & 0.05 \\
%		\hline 
%		CBL & 11.88 & 0.05 & 13.11 & 0.04\\ 
%		\hline 
%	\end{tabular}
%\end{table}
% *********************************************************************************
%\begin{table}
%	\caption{A nice looking table}
%	\centering
%	\label{table:nice_table}
%	\begin{tabular}{l c c c c}
%		\hline 
%		\multirow{2}{*}{Dental measurement} & \multicolumn{2}{c}{Species I} & \multicolumn{2}{c}{Species II} \\ 
%		\cline{2-5}
%		& mean & SD  & mean & SD  \\ 
%		\hline
%		I1MD & 6.23 & 0.91 & 5.2  & 0.7  \\
%		
%		I1LL & 7.48 & 0.56 & 8.7  & 0.71 \\
%		
%		I2MD & 3.99 & 0.63 & 4.22 & 0.54 \\
%		
%		I2LL & 6.81 & 0.02 & 6.66 & 0.01 \\
%		
%		CMD & 13.47 & 0.09 & 10.55 & 0.05 \\
%		
%		CBL & 11.88 & 0.05 & 13.11 & 0.04\\ 
%		\hline 
%	\end{tabular}
%\end{table}
% *********************************************************************************
%\begin{table}
%	\caption{Even better looking table using booktabs}
%	\centering
%	\label{table:good_table}
%	\begin{tabular}{l c c c c}
%		\toprule
%		\multirow{2}{*}{Dental measurement} & \multicolumn{2}{c}{Species I} & \multicolumn{2}{c}{Species II} \\ 
%		\cmidrule{2-5}
%		& mean & SD  & mean & SD  \\ 
%		\midrule
%		I1MD & 6.23 & 0.91 & 5.2  & 0.7  \\
%		
%		I1LL & 7.48 & 0.56 & 8.7  & 0.71 \\
%		
%		I2MD & 3.99 & 0.63 & 4.22 & 0.54 \\
%		
%		I2LL & 6.81 & 0.02 & 6.66 & 0.01 \\
%		
%		CMD & 13.47 & 0.09 & 10.55 & 0.05 \\
%		
%		CBL & 11.88 & 0.05 & 13.11 & 0.04\\ 
%		\bottomrule
%	\end{tabular}
%\end{table}
% *********************************************************************************
%\begin{table}[htbp]
%	\centering
%	\caption{Primeros 40 coeficientes correspondientes a la Teoría de Nutación IAU-1980.}
%	\begin{tabular}{cccccccccc}
%		\toprule
%		\multicolumn{1}{c}{Coeficiente} & \multicolumn{1}{c}{$p_l$} & \multicolumn{1}{c}{$p_l'$} & \multicolumn{1}{c}{$p_F$} & \multicolumn{1}{c}{$p_D$} & \multicolumn{1}{c}{$p_\Omega$} & \multicolumn{1}{c}{$\Delta \hat{\psi} \left[0.001''\right]$} & \multicolumn{1}{c}{$\Delta \hat{\psi}_T\left[0.001''\right]$} & \multicolumn{1}{c}{$\Delta \hat{\varepsilon}\left[0.001''\right]$} & \multicolumn{1}{c}{$\Delta \hat{\varepsilon}_T\left[0.001''\right]$} \\
%		\midrule
%		1     & 0     & 0     & 0     & 0     & 1     & -1719960 & -1742 & 920250 & 89 \\
%		2     & 0     & 0     & 0     & 0     & 2     & 20620 & 2     & -8950 & 5 \\
%		3     & -2    & 0     & 2     & 0     & 1     & 460   & 0     & -240  & 0 \\
%		4     & 2     & 0     & -2    & 0     & 0     & 110   & 0     & 0     & 0 \\
%		5     & -2    & 0     & 2     & 0     & 2     & -30   & 0     & 10    & 0 \\
%		6     & 1     & -1    & 0     & -1    & 0     & -30   & 0     & 0     & 0 \\
%		7     & 0     & -2    & 2     & -2    & 1     & -20   & 0     & 10    & 0 \\
%		8     & 2     & 0     & -2    & 0     & 1     & 10    & 0     & 0     & 0 \\
%		9     & 0     & 0     & 2     & -2    & 2     & -131870 & -16   & 57360 & -31 \\
%		10    & 0     & 1     & 0     & 0     & 0     & 14260 & -34   & 540   & -1 \\
%		11    & 0     & 1     & 2     & -2    & 2     & -5170 & 12    & 2240  & -6 \\
%		12    & 0     & -1    & 2     & -2    & 2     & 2170  & -5    & -950  & 3 \\
%		13    & 0     & 0     & 2     & -2    & 1     & 1290  & 1     & -700  & 0 \\
%		14    & 2     & 0     & 0     & -2    & 0     & 480   & 0     & 10    & 0 \\
%		15    & 0     & 0     & 2     & -2    & 0     & -220  & 0     & 0     & 0 \\
%		16    & 0     & 2     & 0     & 0     & 0     & 170   & -1    & 0     & 0 \\
%		17    & 0     & 1     & 0     & 0     & 1     & -150  & 0     & 90    & 0 \\
%		18    & 0     & 2     & 2     & -2    & 2     & -160  & 1     & 70    & 0 \\
%		19    & 0     & -1    & 0     & 0     & 1     & -120  & 0     & 60    & 0 \\
%		20    & -2    & 0     & 0     & 2     & 1     & -60   & 0     & 30    & 0 \\
%		21    & 0     & -1    & 2     & -2    & 1     & -50   & 0     & 30    & 0 \\
%		22    & 2     & 0     & 0     & -2    & 1     & 40    & 0     & -20   & 0 \\
%		23    & 0     & 1     & 2     & -2    & 1     & 40    & 0     & -20   & 0 \\
%		24    & 1     & 0     & 0     & -1    & 0     & -40   & 0     & 0     & 0 \\
%		25    & 2     & 1     & 0     & -2    & 0     & 10    & 0     & 0     & 0 \\
%		26    & 0     & 0     & -2    & 2     & 1     & 10    & 0     & 0     & 0 \\
%		27    & 0     & 1     & -2    & 2     & 0     & -10   & 0     & 0     & 0 \\
%		28    & 0     & 1     & 0     & 0     & 2     & 10    & 0     & 0     & 0 \\
%		29    & -1    & 0     & 0     & 1     & 1     & 10    & 0     & 0     & 0 \\
%		30    & 0     & 1     & 2     & -2    & 0     & -10   & 0     & 0     & 0 \\
%		31    & 0     & 0     & 2     & 0     & 2     & -22740 & -2    & 9770  & -5 \\
%		32    & 1     & 0     & 0     & 0     & 0     & 7120  & 1     & -70   & 0 \\
%		33    & 0     & 0     & 2     & 0     & 1     & -3860 & -4    & 2000  & 0 \\
%		34    & 1     & 0     & 2     & 0     & 2     & -3010 & 0     & 1290  & -1 \\
%		35    & 1     & 0     & 0     & -2    & 0     & -1580 & 0     & -10   & 0 \\
%		36    & -1    & 0     & 2     & 0     & 2     & 1230  & 0     & -530  & 0 \\
%		37    & 0     & 0     & 0     & 2     & 0     & 630   & 0     & -20   & 0 \\
%		38    & 1     & 0     & 0     & 0     & 1     & 630   & 1     & -330  & 0 \\
%		39    & -1    & 0     & 0     & 0     & 1     & -580  & -1    & 320   & 0 \\
%		40    & -1    & 0     & 2     & 2     & 2     & -590  & 0     & 260   & 0 \\
%		\bottomrule
%	\end{tabular}%
%	\label{table:coefs_nut}
%\end{table}%
% *********************************************************************************
% \ang{23.43929111}
% \ang{;;46.8150}
% $\pm \SI{0.9}{\second}$
% \SI{1e-22}{}
% ...utilizando mediciones en tierra de $F_{\SI{10.7}{}}$: Al dejar vacío el segundo 
% argumento, lo que hace es usar el formato numérico SI, y no el local. O sea, aparece 
% "10.7" en vez de "10,7".
%
% *********************************************************************************
%% This bit allows you to either specify only the files which you wish to
%% process, or `all' to process all files which you \include.
%% Krishna Sethuraman (1990).
%
%\typein [\files]{Enter file names to process, (chap1,chap2 ...), or `all' to process all files:}
%\def\all{all}
%\ifx\files\all \typeout{Including all files.} \else \typeout{Including only \files.} \includeonly{\files} \fi
% *********************************************************************************
% Ignacio Santiago Husain - 2017.
% *********************************************************************************

%-----------------------------------%
%									%
%		Comienzo del documento		%
%									%
%-----------------------------------%
\begin{document}
%-----------------------------------%
%									%
%			Caratula				%
%									%
%-----------------------------------%
\pagestyle{fancy}
\begin{titlepage}
	\newcommand{\HRule}{\rule{\linewidth}{0.5mm}} % Defines a new command for horizontal lines, change thickness here
	\center % Centre everything on the page
	
	\thispagestyle{empty}
	\begin{center}
		\includegraphics[scale=1]{includes/banner_fiuba.pdf}\\
	\end{center}
	
% 	\vspace*{\stretch{1}}
	
	\textsc{\LARGE \textsc{[66.20] Organización de Computadoras}}\\[0.5cm]
	\textsc{\Large \textsc{Trabajo Práctico 1}}\\[0.5cm]
	\textsc{\large 2\textsuperscript{do} Cuatrimestre 2018}\\[0.25cm]
	
	\HRule\\[0.5cm]
	{\huge\bfseries Conjunto de instrucciones MIPS}\\[0.2cm]
	\HRule\\[0.5cm]
	
	\begin{tabbing}
		\hspace{2cm}\=\+
		\underline{AUTORES}\hspace{-1cm}\=\+\hspace{1cm}\=\hspace{6cm}\=\\[0.2cm]
		
		Husain, Ignacio Santiago.	\>\>- \#90.117\\
		\>\footnotesize{$<$santiago.husain@gmail.com$>$}\\
		
		Pesado, Lucía.	\>\>- \#98.275\\
		\>\footnotesize{$<$luupesado@gmail.com$>$}\\
		
		Verstraeten, Federico.	\>\>- \#92.892\\
		\>\footnotesize{$<$federico.verstraeten@gmail.com$>$}\\
		
		\<\underline{CÁTEDRA}\\[0.2cm]
		Dr. Ing. Hamkalo, José Luis. \\
		
		\<\underline{CURSO}\\[0.2cm]
		Ing. Santi, Leandro. \\
		Ing. Perez Masci, Hernán. \\
		Ing. Natale, Luciano. \\[1cm]
		
		\<\underline{FECHA DE ENTREGA}\>\>\> \today
		\\[0.2cm]
		
		\<\underline{FECHA DE APROBACIÓN}\>\>\> 
		\\[0.2cm]
		
		\<\underline{CALIFICACIÓN}\>\>\> 
		\\[0.2cm]
		
		\<\underline{FIRMA DE APROBACIÓN}
		\\[0.5cm]
		
	\end{tabbing}
	
	
% 	\vspace*{\stretch{1}}
	
	
\end{titlepage}

%-----------------------------------%
%									%
%			Indice					%
%									%
%-----------------------------------%
\clearpage

\tableofcontents							

\clearpage

%-----------------------------------%
%									%
%		Seccion:Enunciado			%
%									%
%-----------------------------------%
\includepdf[pages=1,scale=0.95,pagecommand = \section{Enunciado del trabajo práctico}\label{enunciado},offset=10 -10]{includes/tp1-2018-2q.pdf}
\includepdf[pages={2-last},scale=0.95,pagecommand = {},offset=10 -10]{includes/tp1-2018-2q.pdf}
%-----------------------------------%
%									%
%		Seccion:Objetivos			%
%									%
%-----------------------------------%
\section{Objetivos}
El presente trabajo tiene los siguientes objetivos:
\begin{itemize}
    \item Diseñar un codificador/decodificador (\emph{codec}) de información en formato base 64.
    \item Utilizar el lenguaje \textit{Assembly} MIPS32, para la implementaciiones del codificador y decodificador, desarrollando las funciones \texttt{base64\_encode()} y \texttt{base64\_decode()}, y las auxiliares que fueran necesarias.
    \item Utilizar la ABI presentada por la cátedra para el desarrollo de los códigos en \textit{Assembly}.
    \item Utilizar el lenguaje de programación C para desarrollar lo necesario para el procesamiento de las opciones de línea de comandos, apertura y cierre de archivos, y reporte de errores, e invocar el código en \textit{Assembly}.
    \item Compilar dicho codec en el sistema operativo netBSD en una máquina con arquitectura MIPS32.
    \item Realizar pruebas de caja negra para verificar que el programa está funcionando de manera correcta.
    %\item Utilizar la infraestructura básica que será utilizada en trabajos posteriores para programar en assembly MIPS32.
\end{itemize}
%-------------------------------%
%								%
%			Seccion				%
%								%
%-------------------------------%
\section{Diseño e implementación del programa}

Se diseñó un programa en lenguaje ANSI C que implementa el codec descripto en el documento RFC2045 \emph{Base64 Content-Transfer-Encoding} \cite{rfc2045}, \cite{wiki_base64}, con la diferencia de que si en la etapa de decodificación se encuentra un carácter no válido, la ejecución del programa se suspende, notificando al usuario del error producido.

El programa se estructura de la siguiente manera:
\begin{itemize}
	\item \underline{Análisis gramatical de la línea de comandos}: se analizan las opciones ingresadas por la línea de comandos haciendo uso de la función \texttt{parseCmdline()}. La misma se encarga de inicializar una estructura del tipo \texttt{params\_t} utilizada para almacenar las opciones que ingresó el cliente, y cuya definición es
	\begin{lstlisting}[language=C, style=StyleC]
    	typedef struct params_t
        {
          char *action;
          FILE *inputStream;
          FILE *outputStream;
        } params_t;
	\end{lstlisting}
	Además, hace uso de la función \texttt{getopt\_long()} de la biblioteca \texttt{getopt.h}. Dicha función provee una forma simple de procesar cada opción que es leída, extrayendo los argumentos de cada una. En caso de que no se encuentre alguna opción, se utiliza su valor por defecto según las especificaciones del trabajo.

	\item \underline{Validación de opciones:} a medida que se va analizando cada opción de la línea de comandos, se valida cada una de ellas utilizando las funciones 
	\begin{lstlisting}[language=C, style=StyleC]
    	outputCode optAction()
        outputCode optOutput()
        outputCode optInput()
        void optHelp()
        void optVersion()
    \end{lstlisting}
    Además, las mismas realizan la correcta inicialización de las diferentes variables dentro de la estructura \texttt{params\_t} descripta en el punto anterior, o en caso de que el usuario ingresó las opciones de ayuda e indicación de versión del programa, se imprime por el flujo \texttt{stderr} dicha información.
    
	En caso de que se encuentre algún error en el argumento de alguna de las opciones, el usuario es informado por el flujo \texttt{stderr}, y se aborta la ejecución del programa utilizando la función \texttt{exit()}. Para ello, se creó un tipo enumerativo para simplificar el manejo de errores, definiendo los códigos que pueden devolver las funciones desarrolladas:
	\begin{lstlisting}[language=C, style=StyleC]
	typedef enum outputCodes_ {
	outOK,
	outERROR
	} outputCode;
	\end{lstlisting}
	
	\item \underline{Codificación/Decodificación:} una vez realizada la validación de las opciones y carga de configuración del programa, se utilizan las funciones \texttt{base64\_encode()} o \texttt{base64\_decode()} para codificar o decodificar el flujo de entrada, dependiendo de qué fue lo que el cliente le solicitó al programa.
	
	Internamente, ambas funciones se encargan de leer y escribir sobre los archivos de entrada y salida definidos por sus respectivos \emph{file descriptors}, además de realizar el correspondiente manejo de errores leyendo la variable global \texttt{errno}.
	
	Además, hacen uso de las funciones \texttt{b256To64()} y \texttt{b64To256()} para codificar y decodificar los caracteres que se vayan leyendo.
	
	Se decidió realizar esta división de tareas para que no tener funciones monolíticas con demasiadas responsabilidades (lectura y escritura de archivos, manejo de errores, y codificación y decodificación de caracteres).
	
	\item \underline{Terminación del programa:} una vez finalizada la codificación o decodificación, se verifican si se produjeron errores notificando al usuario mediante el flujo \texttt{stderr}, y se cierran los flujos de entrada y salida, y se retorna al sistema operativo.
\end{itemize}

El código fuente desarrollado en ASM MIPS32 y en C se encuentra en el apéndice (\ref{appendix_codigo_fuente}) con sus respectivos \emph{headers} en (\ref{appendix_headers}).

A continuación se describen las secciones importantes de las funciones desarrolladas y se muestra para cada una de ellas el stack que deben crear según la convención descripta en \cite{abi} y \cite{abi_curso}. Se recuerda que se debe agregar padding para alinear el stack a 8 bytes como se requiere en la convención de llamadas a función, por lo que hay algunos stacks que incorporan las variables con nombre \texttt{PADDING\_X} para mostrar esto. Además, varias de ellas van definiendo su tamaño dependiendo de algunas constantes que se debieron utilizar para parametrizar el problema, por lo que se los calculó en función de ellas para una mejor comprensión.

Por otro lado, debido a que se escribieron aproximadamente 1500 líneas en código ASM, se decidió dividir todo el desarrollo en 8 archivos .S para facilitar la comprensión del lector y también del desarrollo que se hizo. Uno de los problemas que se generarían al tener un solo archivo .S es que habría que definir varias etiquetas con nombres únicos para todos los \emph{branchings} que se realizan. De la misma forma, se decidió crear dos archivos .h; \texttt{base64.h} y \texttt{base64\_s.h} para declarar y definir diferentes funciones y constantes. El primero es utilizado por \texttt{main.c} y define mensajes de error del parser junto con las declaraciones
\begin{lstlisting}[language=C, style=StyleC]
extern const char *errmsg[];
extern int base64_encode(int infd, int outfd);
extern int base64_decode(int infd, int outfd);
\end{lstlisting}
para indicarle al compilador que dichos símbolos están definidos en otros archivos (los .S). El segundo header define todas las constantes importantes que utiliza el codec y los mensajes de error. Además de incluir las directivas de inclusión de los headers
\begin{lstlisting}[language=C, style=StyleC]
#include <mips/regdef.h>
#include <sys/syscall.h>
#include <sys/errno.h>
\end{lstlisting}
para hacer uso de los nombres de los registros de la arquitectura MIPS32, las llamadas a sistema, y del manejo de errores.

Además, fue fundamental dicha división ya que permitió realizar depuraciones más simples cuando tuvimos diferentes errores en la codificación del programa.

Respecto del trabajo práctico anterior, se eliminaron todas las llamadas a funciones de la bilioteca standard de C. En particular, se hacía uso de las funciones \texttt{strlen()} y \texttt{strcat()} en la función de codificación, y de \texttt{getc()} y \texttt{fputs()} para la lectura y escritura de los flujos de entrada y salida.

Por último, para facilitar la comprensión de los archivos .S desarrollados, se incluye dentro del directorio \texttt{\ src\ equivalent\_c\_code} los códigos C equivalentes de las funciones traducidas.

\subsection{\texttt{base64.S}}
Se decidió que el archivo fuente \texttt{base64.S} contenga las definiciones de la tabla de codificación y de los mensajes, ya que son usadas por varios archivos fuentes. El contenido del mismo es el siguiente
\begin{lstlisting}[language=C, style=StyleC]
# include "base64_s.h"

.globl translationTableB64

translationTableB64:
.ascii	"ABCDEFGHIJKLMNOPQRSTUVWXYZabcdefghijklmnopqrstuvwxyz0123"
.ascii	"456789+/"

msg0:
.asciiz	ERROR_OUTPUT_STREAM_WRITING_MSG

msg1:
.asciiz	ERROR_INPUT_STREAM_READING_MSG
msg2:
.asciiz	ERROR_B64_CHAR_NOT_FOUND_MSG
.globl	errmsg
.data
errmsg:
.word	0
.word	msg0
.word	msg1
.word	msg2
\end{lstlisting}
donde puede verse que se utilizaron las directivas \texttt{.globl} para definir dichos arreglos globales, y así hacerlos visibles a todas las funciones que las requieran.

\subsection{\texttt{base64\_encode()}}

Se encarga de leer caracteres del archivo de entrada y volcarlos en un buffer de salida (descripto más adelante). La misma realiza la llamada a sistema \texttt{SYS\_READ} para leer los caracteres, como se muestra en el siguiente fragmento de código
\begin{lstlisting}[language=C, style=StyleC]
# Read from the input stream.
	lw	  a0,infd($fp)
	addu  a1,$fp,inChar
	lw	  a2,nbytes($fp)
	li    v0, SYS_read
	syscall
# Store the syscall return value.
	sw	  v0,bytesRead($fp)
\end{lstlisting}
donde se guarda el resultado en la variable \texttt{inChar}.

Para la verificación de errores, luego de cada llamada de lectura, se lee la variable global \texttt{errno}, como se muestra a continuación
\begin{lstlisting}[language=C, style=StyleC]
# Check if there were write errors.
  lw  	t0,errno
  sw	t0,errsv($fp)
  beq	t0,zero,no_read_errors
  li	v0,ERROR_NUMBER_INPUT_STREAM_READING_MSG
  b   	exit_function
\end{lstlisting}
y se retorna en caso de que hayan habido errores.

Una vez leído los caracteres, debe realizar el llamado al codificador, cargando correctamente la dirección de memoria del mismo. En el siguiente fragmento se muestra dicha llamada
\begin{lstlisting}[language=C, style=StyleC]
# Call 'b256To64()' to encode input.
	addu	v0,$fp,outBlock
	lbu	    v1,inChar($fp)
	lb	    a2,inputEnded($fp)
	move	a0,v0
	move    a1,v1
	la	    t9,b256To64
	jal	    ra,t9
# Save encoder output.
	sb	    v0,encodedCharsCount($fp)
\end{lstlisting}

Una vez que se realizó la codificación, se llama a la función \texttt{printChar()} que imprime en el archivo de salida en caso de que el buffer de salida se haya llenado, y si no, se continúa la codificación. Cuando la misma termina porque no quedan más caracteres por codificar, se vacía dicho buffer y se retorna a \texttt{main()}. El fragmento de código utilizado para la llamada es el siguiente
\begin{lstlisting}[language=C, style=StyleC]
# Call 'printChar()'.
	addu	a0,$fp,outputBuffer
	addu	t0,$fp,outBlock
	lw	    t1,index1($fp)
	addu	t0,t0,t1
# We are loading a byte only, not the entire word.
	lbu	    t0,0(t0)
	move	a1,t0
	la	    t9,printChar
	jal	    ra,t9
\end{lstlisting}

El stack que crea la función se muestra en la figura \ref{stack_base64_encode}. El código se encuentra en el apéndice \ref{app_base64_encode}.

\begin{table}[H]
	\centering
	\begin{tabular}{@{}l|c|@{}}
		\cmidrule(l){2-2} 
		\multicolumn{1}{c|}{\textbf{}} & \textbf{Argument-building area (ABA)} \\ \cmidrule(l){2-2} 
		00 & ARG0 \\ \cmidrule(l){2-2} 
		04 & ARG1 \\ \cmidrule(l){2-2} 
		08 & ARG2 \\ \cmidrule(l){2-2}
		12 & ARG3 \\ \cmidrule(l){2-2} 
		\multicolumn{1}{c|}{\textbf{}} & \textbf{Local and Temporary Area (LTA)} \\ \cmidrule(l){2-2} 
		16 & inChar \\ \cmidrule(l){2-2} 
		17 & totalEncodedCharsCount \\ \cmidrule(l){2-2} 
		18 & encodedCharsCount \\ \cmidrule(l){2-2} 
		19 & inputEnded \\ \cmidrule(l){2-2} 		
		20 & bytesRead \\ \cmidrule(l){2-2} 		
		24 & nbytes \\ \cmidrule(l){2-2} 		
		28 & nOutputBlock \\ \cmidrule(l){2-2} 		
		32 & index1 \\ \cmidrule(l){2-2} 				 
		36 & errsv \\ \cmidrule(l){2-2} 		
		40 & outputBuffer \\ \cmidrule(l){2-2} 		
		44 + \\BUFFER\_SIZE +\\ BUFFER\_ARRAY\_OFFSET & outBlock \\ \cmidrule(l){2-2} 		
		outBlock + 4 & PADDING\_1 \\ \cmidrule(l){2-2} 		
		\multicolumn{1}{c|}{\textbf{}} & \textbf{Saved-registers area (SRA)} \\ \cmidrule(l){2-2} 
		PADDING\_1 + 4 & GP \\ \cmidrule(l){2-2} 
		GP + 4 & FP \\ \cmidrule(l){2-2} 
		FP + 4 & RA \\ \cmidrule(l){2-2} 
		RA + 4 & infd \\ \cmidrule(l){2-2} 
		infd + 4 & outfd \\ \cmidrule(l){2-2} 
		outfd + 4 & PADDING\_2 \\ \cmidrule(l){2-2} 
	\end{tabular}
	\caption{Diseño del stack de \texttt{base64\_encode()}. La primer columna se corresponde con el offset en bytes respecto del frame pointer.}
	\label{stack_base64_encode}
\end{table}

donde las constantes 
\begin{lstlisting}[language=C, style=StyleC]
BUFFER_SIZE
BUFFER_ARRAY_OFFSET
\end{lstlisting}
están definidas en \texttt{base64\_s.h}.

El tamaño del stack queda definido por la directiva
\begin{lstlisting}[language=C, style=StyleC]
#define SIZE PADDING\_2 + 4
\end{lstlisting}

\subsection{\texttt{b256To64()}}

Se tuvo que tener cuidado en la codificación ya que se hacen manejos de a bytes y no de words, por lo que se utilizaron instrucciones del tipo \texttt{lbu} y \texttt{sb}. Además, también hubo que tener cuidado por las extensiones de signo que realizan algunas instrucciones. Por ejemplo, si había que comparar una constante con algún byte, la instrucción \texttt{li} realiza extensión de signo, y los resultados de las comparaciones entre bytes y words requerían que solo se esté comparando el byte menos significativo. Estos recaudos se debieron a que en un principio el programa no pasaba los tests desarrollados, por lo que luego de analizar detenidamente con \texttt{gdb} y algunas secciones del código generado por \texttt{GCC}, se pudo comprender qué era lo que estaba sucediendo. El error más común fue el de olvidarnos de la extensión de signo realizada por algunas instrucciones, produciendo comparaciones entre bytes erróneas.

Para el acceso a la tabla de codificación, se utilizó la instrucción \texttt{lbu}, como se muestra en el siguiente fragmento.
\begin{lstlisting}[language=C, style=StyleC]
# Set outBlock[0].
  lbu	   t0,headByte($fp)
  lbu	   t0,translationTableB64(t0)
  lw	   t1,outBlock($fp)
# We only need the LSB.
  sb	   t0,0(t1)
\end{lstlisting}

El stack que crea la función se muestra en la figura \ref{stack_b256To64}. El código se encuentra en el apéndice \ref{app_b256To64}.
	
	\begin{table}[H]
		\centering
		\begin{tabular}{@{}l|c|@{}}
			\cmidrule(l){2-2} 
			\multicolumn{1}{c|}{\textbf{}} & \textbf{Argument-building area (ABA)} \\ \cmidrule(l){2-2} 
			00 & ARG0 \\ \cmidrule(l){2-2} 
			04 & ARG1 \\ \cmidrule(l){2-2} 
			08 & ARG2 \\ \cmidrule(l){2-2}
			12 & ARG3 \\ \cmidrule(l){2-2} 
			\multicolumn{1}{c|}{\textbf{}} & \textbf{Local and Temporary Area (LTA)} \\ \cmidrule(l){2-2} 
			16 & headByte \\ \cmidrule(l){2-2} 
			17 & prevByte \\ \cmidrule(l){2-2} 
			18 & encodedCharsCount \\ \cmidrule(l){2-2} 
			20 & i \\ \cmidrule(l){2-2} 
			\multicolumn{1}{c|}{\textbf{}} & \textbf{Saved-registers area (SRA)} \\ \cmidrule(l){2-2} 
			24 & GP \\ \cmidrule(l){2-2} 
			28 & FP \\ \cmidrule(l){2-2} 
			32 & RA \\ \cmidrule(l){2-2} 
			36 & outBlock \\ \cmidrule(l){2-2} 
			40 & inChar \\ \cmidrule(l){2-2} 
			41 & inputEnded \\ \cmidrule(l){2-2} 
			inputEnded + 3 & PADDING\_0 \\ \cmidrule(l){2-2} 
		\end{tabular}
		\caption{Diseño del stack de \texttt{b256To64()}. La primer columna se corresponde con el offset en bytes respecto del frame pointer.}
		\label{stack_b256To64}
	\end{table}
	
	El tamaño del stack queda definido por la directiva
	\begin{lstlisting}[language=C, style=StyleC]
	#define SIZE_b256To64 PADDING_0 + 4
	\end{lstlisting}
	
\subsection{\texttt{addPadding()}}

Se desarrolló la función \texttt{addPadding()} para agregar el padding cuando hiciera falta en el proceso de codificación. Además, para reducir en algunas líneas el tamaño de los archivos fuente y poder depurar mejor el programa en caso de errores.

El stack que crea la función se muestra en la figura \ref{stack_addPadding}. El código se encuentra en el apéndice \ref{app_addPadding}.

\begin{table}[H]
	\centering
	\begin{tabular}{@{}l|c|@{}}
		\cmidrule(l){2-2} 
		\multicolumn{1}{c|}{\textbf{}} & \textbf{Argument-building area (ABA)} \\ \cmidrule(l){2-2} 
		00 & ARG0 \\ \cmidrule(l){2-2} 
		04 & ARG1 \\ \cmidrule(l){2-2} 
		08 & ARG2 \\ \cmidrule(l){2-2}
		12 & ARG3 \\ \cmidrule(l){2-2} 
		\multicolumn{1}{c|}{\textbf{}} & \textbf{Local and Temporary Area (LTA)} \\ \cmidrule(l){2-2} 
		16 & i \\ \cmidrule(l){2-2} 
		\multicolumn{1}{c|}{\textbf{}} & \textbf{Saved-registers area (SRA)} \\ \cmidrule(l){2-2} 
		20 & RA \\ \cmidrule(l){2-2} 
		24 & FP \\ \cmidrule(l){2-2} 
		28 & GP \\ \cmidrule(l){2-2} 
		32 & outBlock \\ \cmidrule(l){2-2} 
	    36 & pad \\ \cmidrule(l){2-2} 
	    pad + 4 & numberPad \\ \cmidrule(l){2-2} 
	    numberPad + 4 & PADDING\_0 \\ \cmidrule(l){2-2} 
	\end{tabular}
	\caption{Diseño del stack de \texttt{addPadding()}. La primer columna se corresponde con el offset en bytes respecto del frame pointer.}
	\label{stack_addPadding}
\end{table}

El tamaño del stack queda definido por la directiva
\begin{lstlisting}[language=C, style=StyleC]
#define SIZE PADDING_0 + 4
\end{lstlisting}

\subsection{\texttt{base64\_decode()}}

\texttt{base64\_decode()} se encarga de decodificar el flujo de entrada. La función es similar a \texttt{base64\_encode()}, donde también se realizan llamadas a sistema para leer el flujo de caracteres de entrada.

El stack que crea la función se muestra en la figura \ref{stack_base64_decode}. El código se encuentra en el apéndice \ref{app_base64_decode}.
	
	\begin{table}[H]
		\centering
		\begin{tabular}{@{}l|c|@{}}
			\cmidrule(l){2-2} 
			\multicolumn{1}{c|}{\textbf{}} & \textbf{Argument-building area (ABA)} \\ \cmidrule(l){2-2} 
			00 & ARG0 \\ \cmidrule(l){2-2} 
			04 & ARG1 \\ \cmidrule(l){2-2} 
			08 & ARG2 \\ \cmidrule(l){2-2}
			12 & ARG3 \\ \cmidrule(l){2-2} 
			\multicolumn{1}{c|}{\textbf{}} & \textbf{Local and Temporary Area (LTA)} \\ \cmidrule(l){2-2} 
			16 & readChar \\ \cmidrule(l){2-2} 
			readChar + B64\_CHARS\_PER\_BLOCK & inBlock \\ \cmidrule(l){2-2} 
			inBlock + OUTPUT\_BLOCK\_SIZE + 1 & outBlock \\ \cmidrule(l){2-2} 
			outBlock + 4 & index1 \\ \cmidrule(l){2-2} 		
			index1 + 1 & index2 \\ \cmidrule(l){2-2} 		
			index2 + 1 & decodedCharsCount \\ \cmidrule(l){2-2} 		
			decodedCharsCount + 2 & decodingState \\ \cmidrule(l){2-2} 		
			decodingState + 4 & bytesRead \\ \cmidrule(l){2-2} 				 
			bytesRead + 4 & nbytes \\ \cmidrule(l){2-2} 		
			nbytes + 4 & errsv \\ \cmidrule(l){2-2} 		
			errsv + 4  & outputBuffer \\ \cmidrule(l){2-2} 			
			\multicolumn{1}{c|}{\textbf{}} & \textbf{Saved-registers area (SRA)} \\ \cmidrule(l){2-2} 
			outputBuffer +\\ BUFFER\_SIZE +\\ BUFFER\_ARRAY\_OFFSET & GP \\ \cmidrule(l){2-2} 
			GP + 4 & FP \\ \cmidrule(l){2-2} 
			FP + 4 & RA \\ \cmidrule(l){2-2} 
			RA + 4 & infd \\ \cmidrule(l){2-2} 
			infd + 4 & outfd \\ \cmidrule(l){2-2} 
		\end{tabular}
		\caption{Diseño del stack de \texttt{base64\_decode()}. La primer columna se corresponde con el offset en bytes respecto del frame pointer.}
		\label{stack_base64_decode}
	\end{table}
	
	donde las constantes 
	\begin{lstlisting}[language=C, style=StyleC]
	B64_CHARS_PER_BLOCK
	OUTPUT_BLOCK_SIZE
	BUFFER_SIZE
	BUFFER_ARRAY_OFFSET
	\end{lstlisting}
	están definidas en \texttt{base64\_s.h}.
	
	El tamaño del stack queda definido por la directiva
	\begin{lstlisting}[language=C, style=StyleC]
	#define SIZE outfd + 4
	\end{lstlisting}
	
\subsection{\texttt{b64To256()}}

\texttt{b64To256()} se encarga de decodificar los caracteres de entrada y también es similar a su contrapartida para codificación, donde se tuvieron los mismos recaudos con el manejo de bytes.

El stack que crea la función se muestra en la figura \ref{stack_b64To256}. El código se encuentra en el apéndice \ref{app_b64To256}.

\begin{table}[H]
  \centering
  \begin{tabular}{@{}l|c|@{}}
    \cmidrule(l){2-2} 
    \multicolumn{1}{c|}{\textbf{}} & \textbf{Argument-building area (ABA)} \\ \cmidrule(l){2-2} 
    00 & ARG0 \\ \cmidrule(l){2-2} 
    04 & ARG1 \\ \cmidrule(l){2-2} 
    08 & ARG2 \\ \cmidrule(l){2-2}
    12 & ARG3 \\ \cmidrule(l){2-2} 
    \multicolumn{1}{c|}{\textbf{}} & \textbf{Local and Temporary Area (LTA)} \\ \cmidrule(l){2-2} 
    16 & bitMask \\ \cmidrule(l){2-2} 
    20 & index1 \\ \cmidrule(l){2-2} 
    21 & index2 \\ \cmidrule(l){2-2} 
    22 & accumBit \\ \cmidrule(l){2-2} 
    24 & indexTable \\ \cmidrule(l){2-2} 
    B64\_CHARS\_PER\_BLOCK + 24 & charHolder \\ \cmidrule(l){2-2}
    charHolder + 4 & bitPattern \\ \cmidrule(l){2-2}
    \multicolumn{1}{c|}{\textbf{}} & \textbf{Saved-registers area (SRA)} \\ \cmidrule(l){2-2} 
    bitPattern + 4 & RA \\ \cmidrule(l){2-2} 
    GP + 3 & FP \\ \cmidrule(l){2-2} 
    FP + 4 & GP \\ \cmidrule(l){2-2} 
    \multicolumn{1}{c|}{\textbf{}} & \textbf{Arguments of function} \\ \cmidrule(l){2-2} 
    RA + 4 & outBlock \\ \cmidrule(l){2-2} 
    outBlock + 4 & inBlock \\ \cmidrule(l){2-2}
    inBlock + 4 & decCount \\ \cmidrule(l){2-2}
    decCount + 4 & PADDING\_0 \\ \cmidrule(l){2-2} 
  \end{tabular}
  \caption{Diseño del stack de \texttt{b64To256()}. La primer columna se corresponde con el offset en bytes respecto del frame pointer.}
  \label{stack_b64To256}
\end{table}

donde la constante
\begin{lstlisting}[language=C, style=StyleC]
B64_CHARS_PER_BLOCK
\end{lstlisting}
está definida en \texttt{base64\_s.h}.
	
El tamaño del stack queda definido por la directiva
\begin{lstlisting}[language=C, style=StyleC]
#define SIZE_b64To256 PADDING_0 + 4
\end{lstlisting}	

\subsection{\texttt{printChar()}}

Se decidió implementar un buffer de escritura para no tener que realizar llamadas a sistema cada vez que se necesitaba escribir en el archivo de salida. El código equivalente de dicho buffer es
\begin{lstlisting}[language=C, style=StyleC]
#define BUFFER_IDX_OFFSET 0
#define BUFFER_SIZE_OFFSET 4
#define BUFFER_FILE_DESCRIPTOR_OFFSET 8
#define BUFFER_ARRAY_OFFSET 12

#define BUFFER_SIZE 1024

typedef struct buffer_t
{
  int index;
  int size;
  int fd;
  unsigned char buffer[BUFFER_SIZE];
} buffer_t;
\end{lstlisting}
donde se definieron los offsets dentro de la estructura para ser utilizados en el código ASM y poder acceder a los diferentes campos del mismo.

Para realizar la escritura sobre el archivo de salida, se utiliza la llamada a sistema \texttt{SYS\_WRITE} junto con la lectura de la variable global \texttt{errno} para determinar si existieron errores. A continuación se muestra un fragmento de código donde se muestra dicho procesamiento.
\begin{lstlisting}[language=C, style=StyleC]
  lw    a0, buffer($fp)
  lw    a0, BUFFER_FILE_DESCRIPTOR_OFFSET(a0)
  lw    a1, relativeIndex($fp)
  lw    a2, bytesToWrite($fp)
  li    v0, SYS_write
  syscall
  sw    v0, bytesWriten($fp)

# Check if there were write errors.
  lw    v0,errno
  sw	  v0,errsv($fp)
  beq	  v0,zero,while_loop
  b     exit_function
\end{lstlisting}


El stack que crea la función se muestra en la figura \ref{stack_printChar}. El código se encuentra en el apéndice \ref{app_printChar}.

\begin{table}[H]
	\centering
	\begin{tabular}{@{}l|c|@{}}
		\cmidrule(l){2-2} 
		\multicolumn{1}{c|}{\textbf{}} & \textbf{Argument-building area (ABA)} \\ \cmidrule(l){2-2} 
		00 & ARG0 \\ \cmidrule(l){2-2} 
		04 & ARG1 \\ \cmidrule(l){2-2} 
		08 & ARG2 \\ \cmidrule(l){2-2}
		12 & ARG3 \\ \cmidrule(l){2-2} 
		\multicolumn{1}{c|}{\textbf{}} & \textbf{Local and Temporary Area (LTA)} \\ \cmidrule(l){2-2} 
		16 & errsv \\ \cmidrule(l){2-2} 
		20 & bytesToWrite \\ \cmidrule(l){2-2} 
		24 & bytesWriten \\ \cmidrule(l){2-2} 
		28 & relativeIndex \\ \cmidrule(l){2-2} 		
		\multicolumn{1}{c|}{\textbf{}} & \textbf{Saved-registers area (SRA)} \\ \cmidrule(l){2-2} 
		32 & GP \\ \cmidrule(l){2-2} 
		36 & FP \\ \cmidrule(l){2-2} 
		40 & RA \\ \cmidrule(l){2-2} 
		44 & c \\ \cmidrule(l){2-2} 
		48 & buffer \\ \cmidrule(l){2-2} 
		52 & PADDING\_1 \\ \cmidrule(l){2-2} 
	\end{tabular}
	\caption{Diseño del stack de \texttt{printChar()}. La primer columna se corresponde con el offset en bytes respecto del frame pointer.}
	\label{stack_printChar}
\end{table}

El tamaño del stack queda definido por la directiva
\begin{lstlisting}[language=C, style=StyleC]
#define SIZE PADDING_1 + 4
\end{lstlisting}


\subsection{\texttt{flushBuffer()}}

\texttt{flushBuffer()} es una función similar a \texttt{printChar()}, solo que no realiza la verificación de si el buffer de salida está lleno, ya que justamente lo que se quiere es vaciarlo.

El stack que crea la función se muestra en la figura \ref{stack_flushBuffer}. El código se encuentra en el apéndice \ref{app_flushBuffer}.

\begin{table}[H]
	\centering
	\begin{tabular}{@{}l|c|@{}}
		\cmidrule(l){2-2} 
		\multicolumn{1}{c|}{\textbf{}} & \textbf{Argument-building area (ABA)} \\ \cmidrule(l){2-2} 
		00 & ARG0 \\ \cmidrule(l){2-2} 
		04 & ARG1 \\ \cmidrule(l){2-2} 
		08 & ARG2 \\ \cmidrule(l){2-2}
		12 & ARG3 \\ \cmidrule(l){2-2} 
		\multicolumn{1}{c|}{\textbf{}} & \textbf{Local and Temporary Area (LTA)} \\ \cmidrule(l){2-2} 
		16 & errsv \\ \cmidrule(l){2-2} 
		20 & bytesToWrite \\ \cmidrule(l){2-2} 
		24 & bytesWriten \\ \cmidrule(l){2-2} 
		28 & relativeIndex \\ \cmidrule(l){2-2} 		
		\multicolumn{1}{c|}{\textbf{}} & \textbf{Saved-registers area (SRA)} \\ \cmidrule(l){2-2} 
		32 & GP \\ \cmidrule(l){2-2} 
		36 & FP \\ \cmidrule(l){2-2} 
		40 & RA \\ \cmidrule(l){2-2} 
		44 & buffer \\ \cmidrule(l){2-2} 
	\end{tabular}
	\caption{Diseño del stack de \texttt{flushBuffer()}. La primer columna se corresponde con el offset en bytes respecto del frame pointer.}
	\label{stack_flushBuffer}
\end{table}

El tamaño del stack queda definido por la directiva
\begin{lstlisting}[language=C, style=StyleC]
#define SIZE buffer + 4
\end{lstlisting}

%-------------------------------%
%								%
%			Seccion				%
%								%
%-------------------------------%
\section{Compilación del programa}

Debido al requerimiento de utilizar el programa en una computadora con arquitectura MIPS32, se utiliza el emulador \texttt{GXemul} que provee la cátedra, utilizando una máquina virtual que contiene el sistema operativo NetBSD con las herramientas \texttt{gcc} y \texttt{make} para compilar el programa desarrollado.

Para obtener un ejecutable, se creó un archivo \texttt{makefile} cuyo contenido se puede ver en la sección \ref{app_makefile}. Para ejecutarlo, posicionarse en el directorio \texttt{src/} y ejecutar el siguiente comando:
\begin{lstlisting}[language=bash, style=StyleC]
    $ make
\end{lstlisting}

En el mismo puede verse que se utilizan los archivos .S que contienen el código MIPS32 assembly, que serán utilizados directamente por el assembler. La declaración de los mismos se hace en las variables \texttt{\_SRC1} y \texttt{\_SRC2}:
\begin{lstlisting}[language=bash, style=StyleC]
_SRC1 = main.c base64.S flushBuffer.S printChar.S
_SRC2 = base64_encode.S base64_decode.S b256To64.S b64To256.S addPadding.S
\end{lstlisting}

El programa ejecutable aparecerá en el directorio donde se ejecutó el comando \texttt{make}, con el nombre \texttt{tp1}.

%-------------------------------%
%								%
%			Seccion				%
%								%
%-------------------------------%
\clearpage
\section{Pruebas}

Según Patton en \cite{book_patton}, una prueba de caja negra (o \emph{black-box test}) es aquella donde se realizan pruebas al programa sin acceder al código fuente. Es decir, la única información a la que se tiene acceso es aquella que define el comportamiento del programa según las especificaciones del cliente. En esta sección se muestran los resultados de las distintas pruebas de caja negra que se realizaron sobre el programa para determinar su robustez y fiabilidad ante diferentes tipos de entradas. 

Se creó un script en lenguaje \texttt{Bash} para automatizar las pruebas del programa.  El código del script se encuentra en la sección \ref{app_run_tests}, y está compuesto por 17 test.

La salida del script se divide en 2 secciones, cada una con un encabezado indicando el inicio del nuevo test y su nombre, y varias líneas por cada test. La primer línea del test es el comando ejecutado, indicado con la etiqueta \texttt{Testing}. La segunda indica si el test fue exitoso o no mediante la etiqueta \texttt{PASSED/FAILED} en color verde o rojo respectivamente, y las siguientes líneas son los resultados que produce el programa (mensajes de error, etc...). Por ejemplo, para la prueba de la opción \quotes{-i}, se tiene lo siguiente:
\begin{lstlisting}[language=bash, style=StyleC]
------------------------------------------------------
TEST1: inexistent 'input' stream.
------------------------------------------------------
Testing: ./tp0 -i 1
PASSED 
  PROGRAM OUTPUT:
	ERROR: Can't open input stream.
\end{lstlisting}
donde se ve que el test fue satisfactorio ya que se introdujo un nombre de flujo de entrada inválido.

El script con las pruebas debe ejecutarse dentro en la arquitectura MIPS. Para ello deberán cargarse los códigos fuentes del programa, compilar los mismos, y finalmente posicionarse en el directorio \texttt{src/} para ejecutar el comando \texttt{./runTests.sh} en la terminal provista por el Sistema Operativo \textbf{NetBSD} emulado a través de \textbf{gxemul}.

Si todos los tests pasan, entonces al final de la ejecución se debe obtener el siguiente mensaje.
\begin{lstlisting}[language=bash, style=StyleC]
------------------------------------------------------
Test suite ended.
------------------------------------------------------
  All tests passed.
\end{lstlisting}

En caso de que no pasen todos los tests, el script indicará en color rojo cuántos y cuales de estos no hicieron, similar al siguiente mensaje.

\begin{lstlisting}[language=bash, style=StyleC]
------------------------------------------------------
TEST11: no 'input' option parameters.
------------------------------------------------------
Testing: ./tp0 -i 
FAILED 
  PROGRAM OUTPUT:
	./tp0: option requires an argument -- 'i'

------------------------------------------------------
Test suite ended.
------------------------------------------------------
  Failed tests: 1
\end{lstlisting}


\subsection{Pruebas en las opciones de programa}

En el script de tests se prueban diferentes combinaciones de las opciones de entrada para verificar si el programa es capaz de detectar errores. Los 7 primeros son validaciones utilizando opciones y parámetros inválidos, donde se verifica que al intentar ejecutarlo, el programa termina y retorna un mensaje que indique el motivo de la ejecución fallida. El test \texttt{test4\_valid\_parameters} se corresponde con ejecuciones que retornan un código de éxito. 

Las salidas arrojadas por el script fueron las siguientes:

\lstinputlisting[language=bash, style=StyleC]{tests/output_tests_options.txt}

\subsubsection{Pruebas de codificación y decodificación}

Los tests cuyo nombre son del estilo \texttt{test5x\_IO\_validation} se permiten probar si el codec funciona correctamente, ya sea ingresando palabras y/o archivos cuya codificación es conocida, como también archivos aleatorios que son codificados y decodificados para probar si el programa realiza una operación de identidad. También se probó ingresar al decodificador caracteres que no estén en la tabla de codificación para determinar si el programa cortaba la ejecución como se encuentra especificado en el enunciado.

Además, la prueba \texttt{test57\_IO\_validation} verifica que el programa produzca saltos de línea cada 76 caracteres según la especificación del enunciado, y además que si se codificaron N caracteres, la decodificación produzca N caracteres.

Las salidas del script para dichas pruebas fue la siguiente:

\lstinputlisting[language=bash, style=StyleC]{tests/output_tests_codec.txt}

\clearpage

%-------------------------------%
%								%
%			Seccion				%
%								%
%-------------------------------%
\section{Herramientas de hardware y software utilizadas}\label{sec_tools}

La computadora utilizada para realizar el desarrollo y las pruebas tiene las siguientes especificaciones:
\begin{itemize}
	\item Procesador: Intel i3-6100.
	\item Memoria: 8GB RAM DDR4.
	\item Almacenamiento: Disco magnético de 200GB de 7200RPM.
\end{itemize}

El desarrollo se hizo directamente en el emulador provisto por la cátedra, donde el sistema operativo host fue Linux Ubuntu, cuyos datos de distribución son 
\begin{itemize}
	\item Distributor ID: Ubuntu
	\item Description: Ubuntu 16.04.2 LTS
	\item Release: 16.04
	\item Codename: xenial
\end{itemize}

Además, se utilizaron las siguientes herramientas:
\begin{itemize}
	\item \underline{Compilador del proyecto}: gcc (GCC) 3.3.3 (NetBSD nb3 20040520) \cite{tool_gcc}.
	\item \underline{Control del proceso de compilación}: GNU Make \cite{tool_gnuMake}.
	\item \underline{Compilador del presente informe}: Latex pdfTeX 3.14159265-2.6-1.40.16 (TeX Live 2015/Debian) \cite{tool_latex}.
    \item \underline{Edición de código fuente}: VIM - Vi IMproved 6.4 (2005 Oct 15, compiled May 26 2006 02:17:15) y Atom 1.15.0 \cite{tool_vim} \cite{tool_atom}.
	\item \underline{Depuración del programa}: gdb \cite{tool_gdb}.
\end{itemize}

%-------------------------------%
%								%
%			Seccion				%
%								%
%-------------------------------%
\section{Conclusiones}

Se implementaron en lenguaje \textit{assembly MIPS32} todas las funciones que permiten codificar y decodificar la información en formato \textit{base64}, cumpliendo con las especificaciones del enunciado del trabajo.
Se describieron cada una de las subfunciones que componen las principales, junto con los argumentos que reciben y sus valores de salida. Además, se detalló el \textit{stack frame} generado por cada una de las funciones  y algunas partes de código importantes de la implementación.
Además se describió el modo de compilación del programa utilizando las herramientas \texttt{gcc} y \texttt{make}.

\hfill

Por otro lado, se crearon casos de prueba de caja negra para verificar el correcto funcionamiento del programa. Las mismas cubren test de validación de opciones del programa ingresando combinaciones de opciones válidas e inválidas para determinar si este es capaz de detectar errores. Además, se crearon pruebas de codificación y decodificación sobre archivos de formato entrada conocidos y otros generados de manera aleatoria, y tamaños máximos de líneas, a fin de verificar la robustez del programa.
A diferencia del trabajo anterior, en este no se pudieron evaluar tiempos de ejecución del programa, ya que solo puede ser ejecutado dentro de una arquitectura MIPS que acepte el código \textit{assembly MIPS32}. Como la arquitectura es emulada a través de \texttt{gxemul}, los tiempos medidos estarían sezgados por la performance del emulador y no serían los de la implementación del programa desarrollado.

\hfill

Para finalizar, el trabajo permitió estudiar la arquitectura MIPS32 mediante sus set de instrucciones, entendiendo cómo realiza las diferentes operaciones que uno encuentra en lenguajes de alto nivel. Además se estudió en particular cómo se generan los stacks entre llamadas, aplicando la convención de llamado a funciones de la ABI, y también cómo es que se realiza
una syscall al sistema.

%-------------------------------%
%								%
%			Seccion				%
%								%
%-------------------------------%
\clearpage
\begin{thebibliography}{10}
	\bibitem{book_CompArch} Hennessy, J. L. - Patterson, D. A. - \emph{Computer Architecture: A Quantitative Approach} - 3\textsuperscript{rd} edition - Morgan Kaufmann - 2002.
	\bibitem{book_CompOrg} Patterson, D. A. - Hennessy, J. L. - \emph{Computer Organization and Design: The Hardware/Software Interface} - 3\textsuperscript{rd} edition - Morgan Kaufmann - 2004.
	\bibitem{book_Cprogr} Kernighan, B. W. - Ritchie, D. M. - \emph{C Programming Language} - 2\textsuperscript{nd} edition - Prentice Hall - 1988.
	\bibitem{book_patton} Patton, R. - \emph{Software Testing} - 2\textsuperscript{nd} edition - Sams Indianapolis, IN, USA 2005.
	\bibitem{apuntes} \emph{Apuntes del curso 66.20 Organización de Computadoras} - Cátedra Hamkalo - Facultad de Ingeniería de la Universidad de Buenos Aires.
	\bibitem{abi} \emph{System V Application Binary Interface - MIPS/RISC Processor Supplement} - 3\textsuperscript{nd} edition - 1996.
    \bibitem{abi_curso} Organización de Computadoras - 66.20 \emph{MIPS ABI: Function Call Convention}.
	\bibitem{rfc2045} \emph{RFC 2045} - \hyperlink{make}{https://tools.ietf.org/html/rfc2045\#section-6.8}
	\bibitem{wiki_base64} \emph{Base 64} - \hyperlink{make}{https://en.wikipedia.org/wiki/Base64}
	\bibitem{tool_gnuMake} \emph{GNU Make} - \hyperlink{make}{https://www.gnu.org/software/make/}
	\bibitem{tool_gcc} \emph{GNU Gcc} - \hyperlink{gcc}{https://gcc.gnu.org/}
	\bibitem{tool_latex}
	\href{https://www.latex-project.org/}{\LaTeX - https://www.latex-project.org/}
	\bibitem{tool_vim}
	\href{https://vim.sourceforge.io/}{VIM - https://vim.sourceforge.io/}
	\bibitem{tool_atom}
	\href{https://atom.io/}{Atom - https://atom.io/}
	\bibitem{tool_gdb}
	\href{https://www.gnu.org/software/gdb/}{GNU gdb - https://www.gnu.org/software/gdb/}
\end{thebibliography}
\clearpage

%-----------------------------------%
%									%
%			Seccion:Fuente			%
%									%
%-----------------------------------%
\appendix
\section{Makefile}\label{appendix_makefile}

\subsubsection{makefile}\label{app_makefile}
\lstinputlisting[language=bash, style=StyleC]{src/makefile.make}
\clearpage

\section{Tests}\label{appendix_tests}

\subsubsection{runTests.sh}\label{app_run_tests}
\lstinputlisting[language=bash, style=StyleC]{src/runTests.sh}

\clearpage
\section{Código fuente}\label{appendix_codigo_fuente}

\subsubsection{main.c}\label{app_main}
\lstinputlisting[language=C, style=StyleC]{src/main.c}
\clearpage

\subsubsection{addPadding.S}\label{app_addPadding}
\lstinputlisting[language=C, style=StyleC]{src/addPadding.S}
\clearpage

\subsubsection{b64To256.S}\label{app_b64To256}
\lstinputlisting[language=C, style=StyleC]{src/b64To256.S}
\clearpage

\subsubsection{b256To64.S}\label{app_b256To64}
\lstinputlisting[language=C, style=StyleC]{src/b256To64.S}
\clearpage

\subsubsection{base64.S}\label{app_base64}
\lstinputlisting[language=C, style=StyleC]{src/base64.S}
\clearpage

\subsubsection{base64\_decode.S}\label{app_base64_decode}
\lstinputlisting[language=C, style=StyleC]{src/base64_decode.S}
\clearpage

\subsubsection{base64\_encode.S}\label{app_base64_encode}
\lstinputlisting[language=C, style=StyleC]{src/base64_encode.S}
\clearpage

\subsubsection{flushBuffer.S}\label{app_flushBuffer}
\lstinputlisting[language=C, style=StyleC]{src/flushBuffer.S}
\clearpage

\subsubsection{printChar.S}\label{app_printChar}
\lstinputlisting[language=C, style=StyleC]{src/printChar.S}
\clearpage

\clearpage
\section{Header files}\label{appendix_headers}

\subsubsection{base64\_s.h}\label{app_base64_s_h}
\lstinputlisting[language=C, style=StyleC]{src/base64_s.h}
\clearpage

\subsubsection{base64.h}\label{app_base64_h}
\lstinputlisting[language=C, style=StyleC]{src/base64.h}
\clearpage

\end{document}
